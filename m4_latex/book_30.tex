\documentclass[fleqn]{book}

%% Language and font encodings 
\usepackage[english]{babel} 
\usepackage[utf8x]{inputenc} 
\usepackage[T1]{fontenc} 
\usepackage{fancyhdr} 
%% Sets page size and margins 
\usepackage[a4paper,top=2cm,bottom=2cm,left=3cm,right=2cm,marginparwidth=1cm]{geometry} 
\usepackage{xcolor} 
%% Useful packages 
\usepackage{amsmath} 
\usepackage{graphicx} 
\usepackage[colorinlistoftodos]{todonotes} 
\usepackage[colorlinks=true, allcolors=blue]{hyperref} 
\usepackage{setspace} 
\setcounter{chapter}{6}
\setcounter{section}{4}
\setlength{\headheight}{14pt}
\newcounter{pro1}
\setcounter{pro1}{18}
\newcommand{\pro}{\par\addtocounter{pro1}{1}
\textbf{Problem \arabic{chapter}.\arabic{pro1} }\quad}
\setcounter{equation}{17}
\definecolor{light-gray}{rgb}{0.8,0.8,0.8}
\begin{document} 
\pagestyle{fancy} 
\parindent=0cm

% этим мы убеждаемся, что заголовки глав и 
% разделов используют нижний регистр. 
\renewcommand{\headrulewidth}{0pt}
\fancyhf{} % убираем текущие установки для колонтитулов 


\fancyhead[RO]{\large \textsl {\textbf{113}}} 
\fancyhead[LO]{\large \textsl{6.4 Tangent roots: A daunting transcendental sum}} 
\Large \textrm{The integral, from the
1/D
operator, contributes the  area  under  the ln k curve.   The  correction,
from the 1/2 operator, incorporates the triangular
protrusions (Problem 6.20).  The ellipsis includes
the higher-order corrections (Problem 6.21)—hard
to evaluate using pictures (Problem 4.32) but sim-
ple using Euler–MacLaurin summation (Problem 6.21).} 


\colorbox{light-gray}{
\begin{minipage}{\textwidth}
\large\textrm{\textbf { \pro Integer sums} \\  
Use Euler–MacLaurin summation to find closed forms for the following sums:} \\
\large\textrm{(a) $\sum\limits_{0}^n k^{2}$ (b) $\sum\limits_{0}^n (2k+1)$ (c) $\sum\limits_{0}^n k^{3}.$}
\large\textrm{\textbf{\pro Boundary cases} \\  
In Euler–MacLaurin summation, the constant term is [f(b)+f(a)]/2—one-half of the first term plus one-half of the last term.  The picture for summing ln k (Section 4.5) showed that the protrusions are approximately one-half of the last term, namely ln n.  What, pictorially, happened to one-half of the first term?} \\ 
\\ 
\large\textrm{\textbf{\pro Higher-order terms} \\ 
\\ 
Approximate ln 5! using Euler–MacLaurin summation.} 
\\ 
\large\textrm{\textbf{\pro Basel sum} \\ 
\\ 
The Basel sum $\sum\limits_{1}^{∞} n^{-2}$ may be approximated with pictures (Problem 4.37).} 
\\ 
\large\textrm{However,  the  approximation  is  too  crude  to  help  guess  the  closed  form.   As
Euler did, use Euler–MacLaurin summation to improve the accuracy until you
can confidently guess the closed form. Hint: Sum the first few terms explicitly.} \\
\end{minipage}}
 \section{ \textbf{ Tangent roots: A daunting transcendental sum}}
 \large\textrm {Our farewell example, chosen because its analysis combines diverse street-
fighting tools, is a difficult infinite sum.} \\ 
\large\textrm {Find $S\equiv\sum x_{n}^{-2}$ where the $x_{n}$ are the positive solutions of tan x=x.} \\ 
\Large \textrm{The solutions to tan x=x or, equivalently, the roots of tan x−x, are transcendental and have no closed form, yet a closed form is required for
almost every summation method.  Street-fighting methods will come to our rescue.} \\ 
\\
%****************************************************************************************** 
%****************************************************************************************** 
\newpage 
\pagestyle{fancy} 
% этим мы убеждаемся, что заголовки глав и 
% разделов используют нижний регистр. 
\renewcommand{\headrulewidth}{0pt} 
\fancyhf{} % убираем текущие установки для колонтитулов 

\fancyhead[LE]{\large \textsl {\textbf{114}}} 
\fancyhead[RE]{\large \textsl{6 Analogy}} 
 \textbf{6.4.1 Pictures and easy cases}

\Large\textrm{Begin the analysis with a hopefully easy case.}\\ 
\textsl{What is the first root x1?}\\
\Large\textrm{The roots of tan x−x are given by the intersections  of y=x and y=tan x.Surprisingly, no intersection occurs in the branch of tan x where $0 < x < \pi/2$ (Problem 6.23); the first intersection is just before the asymptote at $x=3\pi/2$. Thus, $x_{1}\approx3\pi/2$}\\
\colorbox{light-gray}
{
\begin{minipage}{\textwidth}
\large\textrm{ \textbf{ \textbf\pro No intersection with the main branch}\\ 
Show symbolically that tan x=x has no solution for $0 < x < \pi/2$. (The result looks plausible pictorially but is worth checking in order to draw the picture.)}\\ 
\end{minipage}
}
\textsl{Where, approximately, are the subsequent intersections?}\\
\Large\textrm{As x grows,  the y=x line  intersects  the y=tan x graph  ever  higher and therefore ever closer to the vertical asymptotes.  Therefore, make the
following asymptote approximation for the big part of $x_{n}$:}\\ 
\begin{equation}
x_{n}\approx (n+\frac{1}{2})\pi.
\end{equation}
\textbf{6.4.2  Taking out the big part}
 
\Large\textrm{This approximate, low-entropy expression for $x_{n}$ gives the big part of S (the zeroth approximation).}

\begin{equation}
S\approx \sum[\underbrace{(n+\frac{1}{2})\pi}_{\approx x_{n}}]^{-2}=\frac{4}{\pi^{2}} \sum_{1}^∞ \frac{1}{(2n+1)^2}.
\end{equation}
\Large\textrm{The sum $\sum_{1}^∞ (2n+1)^{-2}$ is,  from  a  picture  (Section  4.5)  or  from  Euler–
MacLaurin summation (Section 6.3.2), roughly the following integral.}\\ 

\begin{equation}
\sum\limits_{1}^∞ (2n+1)^{-2}\approx\int_{1}^{∞}(2n+1)^{-2}dn=-\frac{1}{2}×\frac{1}{2n+1}\bigg|_1^∞=\frac{1}{6}.
\end{equation}
%************************************************************************************* 
%*********************************************************************************** 
\newpage 
\pagestyle{fancy} 
% этим мы убеждаемся, что заголовки глав и 
% разделов используют нижний регистр. 
\renewcommand{\headrulewidth}{0pt} 
\fancyhf{} % убираем текущие установки для колонтитулов 

\fancyhead[RO]{\large \textsl {\textbf{115}}} 
\fancyhead[LO]{\large \textsl{6.4  Tangent roots:  A daunting transcendental sum}} 
\Large\textrm{Therefore,} 
\begin{equation}
s\approx \frac{4}{\pi^2} × \frac{1}{6}=0.067547...
\end{equation}
\Large\textrm{The shaded protrusions are roughly triangles,
and they sum to one-half of the first rectangle.
That rectangle has area 1/9 ,so}
\begin{equation}
\sum\limits_{1}^∞ (2n+1)^{-2}\approx\frac{1}{6}+\frac{1}{2}×\frac{1}{9}=\frac{2}{9}.
\end{equation}
\Large\textrm{Therefore, a more accurate estimate of S is} 
\begin{equation}
S\approx\frac{4}{\pi^{2}}×\frac{2}{9}=0.090063...,
\end{equation}
\Large\textrm{which is slightly higher than the first estimate.} 

\textsl{Is the new approximation an overestimate or an underestimate?}  

\Large\textrm{The new approximation is based on two underestimates.  First, the asymp-
tote approximation $x_{n} \approx(n+0.5)\pi$ overestimates each $x_{n}$ and therefore
underestimates the squared reciprocals in the sum $\sum{x_{n}^{-2}}$. Second, after
making the asymptote approximation, the pictorial approximation to the
sum $\sum_{1}^∞ (2n+1)^{-2}$ replaces each protrusion with an inscribed triangle
and thereby underestimates each protrusion (Problem 6.24).}

\colorbox{light-gray}
{
\begin{minipage}{\textwidth}
\large\textrm{ \textbf{ \textbf\pro Picture for the second underestimate}\\ 
Draw a picture of the underestimate in the pictorial approximation}\\ 
\begin{equation}
\sum\limits_{1}^∞ (2n+1)^{-2}\approx\frac{1}{6}+\frac{1}{2}×\frac{1}{9}.
\end{equation}
\end{minipage}
}

\textsl{How can these two underestimates be remedied?}

\Large\textrm{The second underestimate (the protrusions) is eliminated by summing $\sum_{1}^∞ (2n+1)^{-2}$ exactly.  The sum is unfamiliar partly because its first term is the fraction 1/9 — whose arbitrariness increases the entropy of the sum. Including the n=0 term, which is 1, and the even squared reciprocals 1 /$(2n)^2$ produces a compact and familiar lower-entropy sum.}
\begin{equation}
\sum\limits_{1}^∞ (2n+1)^{2}+1+\sum\limits_{1}^∞ \frac{1}{(2n)^{2}}=\sum\limits_{1}^∞ \frac{1}{n^{2}}.
\end{equation}
%************************************************************************************* 
%*********************************************************************************** 
\newpage 
\pagestyle{fancy} 
% этим мы убеждаемся, что заголовки глав и 
% разделов используют нижний регистр. 
\renewcommand{\headrulewidth}{0pt} 
\fancyhf{} % убираем текущие установки для колонтитулов 

\fancyhead[LE]{\large \textsl {\textbf{116}}} 
\fancyhead[RE]{\large \textsl{6 Analogy}} 
\Large\textrm{The final, low-entropy sum is the famous Basel sum (high-entropy results are not often famous).  Its value is $B = \pi^{2}/6$ (Problem 6.22).} 

\textsl{How does knowing $B = \pi^{2}/6$ help evaluate the original sum $\sum_{1}^∞ (2n+1)^{-2}$?} \\ 
\Large\textrm{The major modification from the original sum was to include the even
squared reciprocals.  Their sum is B/4.}
\begin{equation}
\sum_{1}^∞ \frac{1}{(2n)^2} = \frac{1}{4}\sum_{1}^∞ \frac{1}{n^2}.
\end{equation}
The second modification was to include the n=0 term. Thus, to obtain, $\sum_{1}^∞ (2n+1)^{-2}$ adjust the Basel value B by subtracting B/4 and then the n=0 term.  The result, after substituting $B=\pi^2/6$ ,is
\begin{equation}
\sum_{1}^∞ \frac{1}{(2n+1)^2} = B− \frac {1}{4}B−1= \frac{\pi^2}{8}−1.
\end{equation}
\Large\textrm{This exact sum, based on the asymptote approximation for $x_{n}$, produces the following estimate of S.} \\ 
\begin{equation}
S\approx \frac{4}{\pi^2} \sum_{1}^∞ \frac{1}{(2n+1)^2} = \frac{4}{\pi^2}(\frac{\pi^2}{8}−1).
\end{equation}
\Large\textrm{Simplifying by expanding the product gives}
\begin{equation}
S\approx \frac{1}{2}-\frac{4}{\pi^2}=0.094715..
\end{equation}
\\
\colorbox{light-gray}
{
\begin{minipage}{\textwidth}
\large\textrm{ \textbf{ \textbf\pro Check the earlier reasoning}\\ 
Check  the  earlier  pictorial  reasoning  (Problem  6.24)  that 1/6+1/18=2/9 underestimates $\sum_{1}^∞ (2n+1)^{-2}$ . How accurate was that estimate?}\\ 
\end{minipage}
}
\Large\textrm{This  estimate  of S is  the  third  that  uses  the  asymptote  approximation $x_{n}\approx(n+0.5)\pi$}. Assembled together, the estimates are
\[
 S\approx \left\{
 \begin{matrix} 
   0.067547 & (\textrm{integral  approximation  to } \sum_{1}^∞ (2n+1)^{-2}),
 \quad  &\\ 0.090063 & (\textrm {integral approximation and triangular overshoots}),
 
 \quad  &\\ 0.090063 & (\textrm {exact sum of} \sum_{1}^∞ (2n+1)^{-2}) .& 
 \end{matrix}
 \right. 
\]
\Large\textrm{Because the third estimate incorporated the exact value of $\sum\limits_{1}^∞ (2n+1)^{-2}$, any remaining error in the estimate of S must belong to the asymptote
approximation itself.} \\
\end{document}
